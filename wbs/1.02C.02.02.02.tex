\paragraph*{1.02C.02.02.02: System Architecture Oversight}
\label{wbs:1.02C.02.02.02}

This WBS element includes all activities related to ensuring that the
constructed LSST Data Management System, including the computing and storage
systems, the processing systems, and the science pipelines, adheres to its
architectural principles and standards and that the Data Management development
processes are followed.  It involves tracking software development; leading,
advising, and educating during design, code, sprint, and other reviews;
contributing to the completeness of verification testing; maintaining the DM
Risk Register; and communicating the DMS architecture internally and
externally.  This WBS element also involves making decisions on design and
process changes to ensure emergent properties of the system such as usability,
reliability, understandability, and maintainability.  The Architecture Team
provides input to decision-making personnel and bodies but does not supervise,
directly control, or exercise a veto over development work except where
explicitly delegated that role.  One such delegation is the Release Manager
role which oversees and coordinates the preparation for each software release.
Architecture Team input about low-level code is conveyed to individual
developers during reviews.  Input about refinement of designs is conveyed to
technical leads and the NCSA Steering Committee.  Input about revisions to
designs or plans is conveyed to technical managers and the NCSA Steering
Committee for incorporation into prioritization.  Interactions with LSST System
Engineering, Operations Planning, Risk Management, and Change Control are
contained within this WBS, as is architectural representation in the DM Systems
Engineering Team and Change Control Board.
