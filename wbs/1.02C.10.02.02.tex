\paragraph*{1.02C.10.02.02: Science Platform Notebook Environment for QA, Commissioning \& User Science}

\begin{description}

\item[1.02C.10.02.02.01: Jupyter Notebook \& Templates]
  A set of notebooks, and templates for making them, that demonstrate key
  features of the capabilities of the system.

\item[1.02C.10.02.02.02: JupyterLab Deployment]
  Architecture, orchestration and deployment configuration for the Science
  Platform Notebook service for commissioning.

\item[1.02C.10.02.02.03: Custom Portals/Notebooks]
  \label{wbs:1.02C.10.02.02.03}
  This WBS element covers supporting the portals delivered by the SUIT team
  (\hyperref[wbs:1.02C.05.08]{1.02C.05.07}) post-delivery where they relate to
  QA and commissioning activities as necessary.

\item[1.02C.10.02.02.04: Notebook Software Environments]
  \label{wbs:1.02C.10.02.02.04}
  Production of environments (e.g. containers) suitable for the execution of
  \hyperref[wbs:1.02C.10.02.02.03]{custom portals/notebooks}.

\item[1.02C.10.02.02.05: Notebook Execution]
  The process to scale notebook execution so they can execute over a large
  dataset. This involves an interface to the batch workflow system.

\item[1.02C.10.02.02.06: Dataspace packaging]
  The packaging and configuration required to deploy the dataspace on a
  platform that is design-matched to the compute and filespace elements of the
  Archive Center dataspace (e.g. if the DAC compute is based on an OpenStack
  architecture, the deliverable of this WBS are the packages, configuration,
  automation deployment and instructions that would allow a Data Access Center
  at an international partner to deploy a Dataspace service on top of their
  open OpenStack compute for their own users).

\end{description}
