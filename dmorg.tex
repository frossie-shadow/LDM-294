\section{Data Management Organization Structure}

This section defines the organization structure for the period in which the DM System is developed and commissioned, up to the start of LSST Observatory operations.

The DM Project Manager (William O'Mullane), Deputy Project Manger (John Swinbank) and DM Project Scientist (Mario Juri\'c), who are known collectively as DM Management, lead the DM Subsystem.
The Project Manager has direct responsibility for coordination with the overall LSST Project Office, the LSST Change Control Board, the LSST Corporation, and LSST partner organizations on all budgetary, schedule, and resource matters.
The Project Scientist has primary scientific and technical responsibility in the DM and responsibility for ensuring that the scientific requirements of the LSST are supported, and is a member on the LSST Project Science Team (PST).

As shown in \figref{fig:dmorg}, the organization now features  major products  each with a product owner
relating to a major element of the DM Subsystem (Level 2 Work Breakdown Structure elements).

\begin{figure}[htbp]
\begin{center}
 \includegraphics[width=\textwidth]{images/DmOrg}
\caption{DM organization with Scientists in Green. \label{fig:dmorg}}
\end{center}
\end{figure}
%figure wom


\subsection {Meetings } \label{sect:meetings}
As a diverse and distributed organization DM staff will participate in a considerable number of meetings.
NSF and Aura have many rules on meeting attendance and LSST keep policies updated accordingly in \citeds{LPM-191} and \citeds{Document-13760}. This includes the travel summary report template \citedsp{Document-13762} every traveler must fill after attending a meeting.

A detailed debrief note or presentation may be asked of travelers to specific meetings of interest by the DMLT.

\subsection {Working Groups } \label{sect:wgs}
Some issues in development of a system like Data Management require more effort to remove than a simple RFC. When t
he decision making process (\appref{sect:ddmp}) can not come to a conclusion the DM PM reserves the right to create
 a short lived working group to deal with the issue. A working group will be given a specific narrow charge, it will be a small group ($\approx 7$ people), it will be time bounded and have a clear deliverable. 
 Members of the group will be agreed by the DMLT to provide the best technical input from all stakeholders perspectives. Members of the working group should discuss in their local organizations and socialize recommendations ahead of adoption. 
 This has been done for the SuperTask for example. 

 \subsection {Studies } \label{sect:studies}
 In some cases DM will initiate studies by external parties to investigate potential alternatives this is especially
  true for technology related activities. 



\subsection {Document Management} \label{sect:docman}

DM documents will follow the Systems Engineering Guidelines of LSST. PDF versions of released documents shall be put in Docushare in accordance with the Project's Document Management Plan \citedsp{LPM-51}. LDM level documents are released on agreement of the DMCCB (\secref{sect:dmccb}), uncontrolled documents such as technotes may be released when the author decides it is appropriate or they are asked to release it by the Project Manager.

The Document Tree for DM is shown in \figref{fig:doctree}, it is not exhaustive but gives a high level orientation for the main documents in DM and how they relate to each other. Some documents shown in red are not yet written.

\begin{figure}
\begin{center}
 \includegraphics[width=0.9\textwidth]{images/DocTree}
\caption{Outline of the documentation tree for DM software relating the high level documents to each other. \label{fig:doctree}}
\end{center}
\end{figure}

\figref{fig:doctree} has one box for End User documentation, this is a major set of documentation for DM which will be web based as  described in  \citeds{LDM-493}. \figref{fig:eudoc} shows the intended web hierarchy for the end user documentation.

\begin{figure}
\begin{center}
 \includegraphics[width=1\textwidth]{images/EndUserDocs}
\caption{Outline of the web hierarchy for the DM end user documentation. \label{fig:eudoc}}
\end{center}
\end{figure}
%Figure from jsick



Service-level documentation follows the layered service architecture of the LSST Data Facility (see \figref{fig:servdoc}).

\begin{figure}
\begin{center}
 \includegraphics[width=0.5\textwidth]{images/servdocs}
\caption{Outline of layered service architecture of the Data Facility. \label{fig:servdoc}}
\end{center}
\end{figure}

\subsubsection {Documentation of Cross-Cutting Aspects for services}

The cross-cutting aspects of the LSST Data Facility, Security and Operational Manageability, are represented by the vertical boxes. Documentation of these aspects describes policies, procedures, and supporting management frameworks, including:
\begin{enumerate}
	\item	LDF service management framework: service catalog, service-level agreements (SLAs), configuration management database (CMDB), service monitoring.
	\item	LDF service management processes and context in the overall project: incident response, request response, issue tracking, problem management and the problem management database, change management and change control authority, release management.
	\item	Overview of the security enclave structure
	\item	Security controls and incident response procedures
	\item	Disaster recovery and continuity policies
\end{enumerate}

\subsubsection{Documentation of Service Layers}

The box at the top of the figure, Use Cases, represents subsystem-level and project-level operational use cases. The next layer, LDF-offered Services, represents specific services offered by the Data Facility which satisfy those use cases. Documentation of this layer includes:

\begin{enumerate}
\item	For each service, a Concept of Operations (ConOps) which summarizes how a service operates to satisfy a use case. The ConOps describes the operational characteristics of the production system, context within overall LSST operations, and representative scenarios. 
\item	For each service, a Theory of Operations, which provides a mental model of a constructed system. The Theory of Operations explains how the constructed service both fulfills the ConOps and integrates with the cross-cutting aspects of the facility. The document describes the overall architecture of the service and dependency on supporting service layers; integration into aspects of computer security, information security and business continuity; and integration into incident reporting and response, availability and capacity management, and change management.
\end{enumerate}

The next two layers, Reusable Production Services and Data, Compute, and IT Security Services, represent tiers of supporting service. Documentation of these layers includes a Theory of Operations, as described above, explaining the dependencies on supporting service and ITC layers, and integration with cross-cutting aspects of the facility.

The ITC box represents hardware components supporting all LDF services. Documentation of ITC describes the system elements at all facility sites, administration within each security enclave and integration with security operations, the overall provisioning plan, ITC system monitoring and integration into the service monitoring framework, and integration into service management processes including configuration management and change management.

The Software box represents service software components being developed by the LSST Data Facility. Documentation of software elements follows the standards of the LSST software stack.

Documents are managed as configuration items in the LSST Data Facility CMDB.

\subsubsection{Draft Documents}

Draft DM documents will be kept in GitHub. A single repository per document will be maintained with the head revision containing the \emph{released } version which should match the version on docushare. Each repository will be included as a \emph{submodule} of a single git repository located at \url{https://github.com/lsst-dm/dm-docs}.

Use of Google Docs or confluence is tolerated but final delivered documents must conform to the standard LSST format, and hence either produced with LaTeX, using the lsst-texmf package\footnote{\url{https://lsst-texmf.lsst.io}}, or Word, using the appropriate LSST template \citedsp{Document-9224, Document-11920}. The precursor document should then be erased with a pointer to the baseline document, stored in GitHub.

End user documentation will most likely and appropriately be web based and the scheme for that is described in \citeds{LDM-493}.

\subsection {Configuration Control} \label{sect:config}

Configuration control of documents is dealt with in \secref{sect:docman}. Here we consider more the operational systems and software configuration control.

\subsubsection{Software Configuration Control}

DM follows a git based versioning system based  on public git repositories and the approach is covered in the developer guide \url{https://developer.lsst.io/}.
The master branch is the stable code with development done in \emph{ticket} branches (named with the id of the corresponding JIRA Ticket describing the work.
Once reviewed a branch is merged to master.\footnote{LSE-14 seem out of date and should be updated or revoked - titled a guideline it seems inappropriate as an LSE.}

As we approach commissioning and operations DM will have a much stricter configuration control.
At this point there will be a version of the software which may need urgent patching, a next candidate release version of the software, and the master.
A patch to the operational version will require the same fix to be made in the two other versions.
The role of the DM Change Control Board (DMCCB; \secref{sect:dmccb}) becomes very important at this point to ensure only essential fixes make it to the live system as patches and that required features are included in planned releases.

We cannot escape the fact that we  will have multiple code branches to maintain in operations which will lead to an increase in work load.
Hence one should consider that perhaps more manpower may be needed in commissioning to cope with urgent software fixes while continuing development.
The other consideration would be that features to be developed post commissioning will probably be delayed more than one may think, as maintenance will take priority.\footnote{WOM identifies this as the maintenance surge.}

\subsubsection{Hardware Configuration Control}

On the hardware side we have multiple configurable items, we need to control which versions of software are on which machines. These days tooling like Puppet make this reasonably painless. Still the configuration  must be carefully controlled to ensure reproducible deployments providing correct and reproducible results. The exact set of released software and other tools on each system should be held in a configuration item list.
Changes to the configuration should be endorsed by the DMCCB.

The sizing model for compute hardware purchasing is detailed in \citeds{LDM-144} \citeds{LDM-141} and \citeds{LDM-138}.

\subsection {Risk Management } \label{sect:risk}

Risks will be dealt with within the LSST Project framework as defined in \citeds{LPM-20}.
Risks in DM may be sent to the DM project manger or Deputy project manager at any time for consideration to be included in the formal risk register (appropriate costed and weighted). All risks are reviewed regularly by the DM Project manager and Systems Engineer (minimum each 3 months).


\subsection {Quality Assurance  } \label{sect:pa}

In accordance with the project QA plan \citeds{LPM-55} we will perform QA on the software products.
This work will mainly be carried out by SQuaRE (\secref{sect:square}).
Quality assurance here means compliance with project guidelines for production, in out case of software production.
A part of this is to have a verification/validation plan(s) which in and of itself is a major task (see \secref{sect:vanv}).


\subsection{Action item control}
Actions in DM are tracked as JIRA issues an periodically reviewed at DMLT meetings.


\subsection {Verification and Validation } \label{sect:vanv}

We intend to verify and validate as much of DM as we can before commissioning and operations.
This will be achieved through testing and operations rehearsals/data challenges.
The verification and validation approach is detailed in \citeds{LDM-503} including a high level test schedule,
the top level schedule is given in \figref{fig:schedule}.
