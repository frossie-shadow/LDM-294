\section{Development Process} \label{sect:devproc}

DM is essentially a large software project; in particular we are developing scientific software with the uncertainties that brings with it.
An Agile process \citep{it:agile} is particularly suited to scientific software development.  The development follows a six month  cyclical approach and  DM  products are under continuous
integration using the Jenkins tool. All code is developed in the GitHub open source repository under an open source license.
Releases follow a six-month cadence but the code on the master branch is intended to be always working with the continuous integration system ensuring this.

How this fits with the Earned Value System is described in \citeds{DMTN-020}.


\subsection{Communications}

The main stories for the six-month planning period are discussed at the DMLT F2F meeting near the beginning of the cycle (See \secref{sect:dmlt}).

The T/CAMs of each of the institutions meet via video on Tuesdays and Fridays for a short \emph{standup} meeting to ensure that any cross-team issues are surfaced and resolved expeditiously.
This meeting is chaired by the Deputy Project Manager.
Each T/CAM notes any significant progress of interest to other teams and any problems or potential problems that may arise.

\subsection{Conventions}
Coding guidelines and conventions are documented online in \url{https://developer.lsst.io}

\subsection{Reviews} \label{sect:reviews}

The DM Project Manager and Subsystem Scientist will periodically convene internal reviews (following \citeds{LSE-159})
of major DM components as necessary to assess progress and maintain the integrity of the overall system. Planned DM reviews will be listed at the LSST Project Review Hub (\url{https://project.lsst.org/reviews/hub/}).
%\begin{itemize}

%\item  Science and Alerts Pipelines Review
%\item   Verification Plan Review
  %\item  Science platform, perhaps in 3 parts
	%\begin{itemize}
	  %\item  JupyterLab
	  %\item SUI portal
	  %\item Web/APIs
	%\end{itemize}
  %\item  Calibration Review
%\end{itemize}

In addition, smaller components of the system will undergo DM-internal design reviews.  The DMPM decides what will be reviewed (with input from all DM members) and is the Decision Making Authority for approving review recommendations.  Participants in the design review will normally include all members of the DMCCB and other experts as appropriate (e.g. the LSST Information Security Officer or designated substitute if there are any security implications).  The design review will check that the design:
\begin{itemize}
\item meets the requirements and satisfies the use cases, and an implementation can be verified as doing so
\item conforms to the LSST DM architecture and has well-defined interfaces
\item is expected to be efficient in terms of labor cost, non-labor cost, and schedule
\item is expected to be reliable, maintainable, supportable, usable, and secure
\item conforms to good engineering practices
\end{itemize}

Design review presentations should include:
\begin{itemize}
\item the identification of the components under review in terms of where they fit within the overall architecture
\item use cases and requirements applicable to the components under review that show how they will be used and how they respond/support all usage
\item an API or other description of the public interfaces to the components under review
\item a description of the internal patterns and algorithms to be used in the design, known limitations to those, and justification why the limitations are acceptable for this development
\item a description of the technological approach to implementation, including use of any third-party components, and reuse of existing elements (e.g. this will be a specialization of the XYZ framework classes)
\item a description of how the function and performance of the component(s) under review will be tested
\end{itemize}
