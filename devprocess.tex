\section{Development Process} \label{sect:devproc}

DM is essentially a large software project, more we are developing scientific software with the in uncertainties that brings with it. 
An agile \citep{it:agile} is particularly suited to scientific software development.  The development follows a six month  cyclical approach and  DM  products are under continuous
integration using the application software Jenkins. All code is developed in the GitHub open source repository under an open source license.
Releases follow a six month cadence but the master is intended to be always working with a continuous integration system ensuring this.

How this fits with the Earned Value System is described in \citeds{DMTN-020}.


\subsection{Communications}

The main stories for the six month planning period are discussed at the DMLT F2F meeting near the beginning of the cycle (See \secref{sect:dmlt}). 

The T/CAMs of each of the institutions meet via video on Tuesdays and Fridays for a short \emph{standup} meeting to ensure that any cross-team issues are surfaced and resolved expeditiously.
This meeting is chaired by the Deputy Project Manager.
Each T/CAM notes any significant progress of interest to other teams and any problems or potential problems that may arise.

\subsection{Conventions}
Coding guidelines and conventions are documented online in \url{https://developer.lsst.io}

\subsection{Reviews} \label{sect:reviews}

The DM Project Manager and Subsystem Scientist will periodically convene internal reviews (following \citeds{LSE-159}) 
of major DM components as necessary to assess progress and maintain the integrity of the overall system. Planned DM reviews will be listed at the LSST Project Review Hub (\url{https://project.lsst.org/reviews/hub/}).
%\begin{itemize}

%\item  Science and Alerts Pipelines Review 
%\item   Verification Plan Review 
  %\item  Science platform, perhaps in 3 parts
	%\begin{itemize}
	  %\item  JupyterLab 
	  %\item SUI portal
	  %\item Web/APIs
	%\end{itemize}
  %\item  Calibration Review 
%\end{itemize}
