\subsection {Document Management} \label{sect:docman}
DM documents will follow the System Engineering Guidelines of LSST. PDF versions of released documents shall be put in Docushare in accordance with the project Document management plan \citell{LPM-51}.

The Document Tree for DM is shown in \figref{fig:doctree}, it is not exhaustive but gives a high level orientation for the main documents in DM and how they relate to each other.

\begin{figure}
\begin{center}
 \includegraphics[width=0.8\textwidth]{images/DocTree}
\caption{Outline of the documentation tree for DM relating the high level documents to each other. \label{fig:doctree}}
\end{center}
\end{figure}


\subsubsection{Draft Documents}
Draft DM documents will be kept in GitHub. A single repository per document will be maintained with the head revision containing the {\em released } version which should match the version on docushare. 

In addition each repository will be included as a {\em submodule} of a single dm-docs git repository namely\url{https://github.com/lsst-dm/dm-docs}. 
\footnote{Use of Google Docs or confluence is tolerated but final delivered documents should look like LSST docs so either done with TeX or Word Templates. The Google doc or Confluence page should then be erased with a pointer to the baseline document. This should be in github.}

An LSST document class fro LaTeX is provided for TeX documents, to use this texmf must be set up to include the texmf folder form the repository. See \url{https://github.com/lsst/lsst-texmf/} to set this up. 

End user documentation will most likely and appropriately be web based and the scheme for that is described in \citeyearpar{LDM-493}.


\subsection {Configuration Control } \label{sect:config}
Configuration control of documents is dealt with in \secref{sect:docman}. Here we consider more the operational systems and software configuration control. 
\subsubsection{Software Configuration Control}
{\bf We should have a configuration management plan covering this.} \\

DM follows a regular gitflow based on public git repositories and the approach is covered in the developer Guide \url{https://developer.lsst.io/}.
The master branch is the stable code with development done in {\em ticket} branches (named with the id of the corresponding Jira Ticket describing the work. Once reviewed a branch is merged to master.\footnote{LSE-14 seem out of date and should be updated or revoked - titled a guideline it seems inappropriate as an LSE.}

As we approach commissioning and operations DM will have a much stricter configuration control. At this point there will be a version of the software which may need urgent patching, a
next candidate release version of the software, and the master. A patch to the operational version
will require the same fix to be made in the two other versions. Th's role of the Configuration Control Board (CCB) becomes very important at this point to ensure only essential fixes make it to the live system as patches and that required features are included in planned releases.


We  can not escape the fact that we  will have multiple code branches to maintain in operations which will lead to an increase in work load.
Hence one should consider that perhaps more manpower may be needed in commissioning to cope with urgent
software fixes while continuing development. The other consideration would be that features to
be developed post commissioning will probably be delayed more than one may think, as maintenance will take priority.\footnote{WOM identifies this as the maintenance surge.}

\subsubsection{Hardware Configuration Control}
On the hardware side we have multiple configurable items, we need to control which versions of software are on which machines. These days tooling like Puppet make this reasonably painless. Still the configuration  must be carefully controlled to ensure reproducible deployments providing correct and reproducible results. The exact set of released software and other tools on each system should be held in a configuration item list. 
Changes to the configuration should be endorsed by the CCB. 


\subsection {Risk Management } \label{sect:risk}

Risks will be dealt with within the project framework defined in \citell{LPM-20}, the specific DM process for risk assessment is defined in \citell{LDM-512}.


\subsection {Quality Assurance  } \label{sect:pa}
In accordance with the project QA plan \citell{LPM-55} ..

\subsection {Verification and Validation } \label{sect:vanv}
We intend to verify and validate as much of DM as we can before commissioning and operations. This will be achieved through testing and
operations rehearsals/data challenges. The Verification and Validation approach is detailed in \citeyearpar{LDM-503}
