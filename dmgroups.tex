\section{Data Management Groups/Bodies} \label{sect:groups}
Since the DM team is distributed in terms of geography and responsibility across the LSST partner and lead institutions, mechanisms are needed to ensure that the project remains on track at all times.  There are three primary coordinating bodies to ensure the management, technical, and quality integrity of the DM project.  All DM institutions have membership on these bodies, and all meet at least once per month during construction and commissioning.

\subsection{DM System Science Team \label{sect:dmsst}
{\bf Mario  please update ..}
The DM Science team is run by DM PS and brings together the DM Subsystem Scientist, the DM System Science Staff (Melissa Graham, Colin Slater), and various Institutional Scientist roles (AP Scientist, DRP Scientist, SUIT Scientist, SQuaRE Scientist, NCSA Science Lead). 
The team works together to define, maintain, and communicate to the DM construction team a coherent vision of the LSST DM system responsive to the overall LSST Project Goals, and ensure the DM System is scientifically validated. In addition the team should  investigate problems an analyse commissioning data. 

\subsection{DM System Engineering Team \label{sect:sysengt}}
The System engineering team is lead by the DM Project Manager and looks after all aspect of system engineering. It is comprised of not only a System Engineer (\secref{role:sysengineer}) but also the Requirements Engineer (\secref{role:reqeng}, Software Architect (\secref{role:softarc}), Operations Architect (\secref{role:opsarc}) and the Pipeline Scientist (\secref{role:pipe}).

While the product owners help DM to create the correct product , fit for purpose, the System engineering team must ensure we do it correctly. This group concerns its self with system wide decisions on architecture and software engineering.  

Within this  group we must:
\begin{itemize}
\item Formalise the Product list/tree for DM, these are not the data products but the DM software and systems which produce the products. 
\item Formalise the documentation tree for DM - which documents need to be produced for each product. 
\item Agree how to trace the baseline requirements verification and validation status.
\item  \ldots
\end{itemize}
 Some of these tasks are obviously delegated tot he individuals in the group. These individuals also are the conduit to the rest of the DM team to raise ideas/issues with the engineering approach. 

 \subsubsection{Communications} 
 The System engineering team will only physically meet to discus specific topics there will not be a regular meeting of the group outside for the one to one meetings with the DM project manager for the individuals in the group. 
Discussions will be held via email until in person talks are required. 

\subsection{DM Leadership Team} \label{sect:dmlt}

The DM Leadership Team (DMLT) purpose is to establish scope of work and resource allocation across DM and ensure overall project management integrity across DM.
The following mandate established the DMLT:

\begin{itemize}
\item Charter/purpose
	\begin{itemize}
	\item Maintain scope of work and keep within resource allocation across DM
	\item Ensure overall project management integrity across DM
	\item Ensure Earned Value management requirements are met
	\end{itemize}
\item Membership
	\begin{itemize}
	\item Co-Chaired by the DM Project Manager  and  DM Project Scientist
	\item Core members are Lead Institution Technical/Control Account Managers (T/CAMs or CAMs)
	\end{itemize}
\item Responsibilities
	\begin{itemize}
	\item Prepares all budgets, schedules, plans
	\item Meets every week to track progress, address issues/risks, adjust work assignments and schedules, and disseminate/discuss general PM communications
	\item Creates and publishes monthly, quarterly, annual progress reports
	\item Meets at start of each software development phase with SAT to establish detailed scope/work plan
	\item Meets with SAT for change control (DMCCB)
	\end{itemize}
\end{itemize}

The DM Leadership Team and the System Engineering Team (\secref{sect:sysengt}) work in synchrony. 
The DMLT makes sure the requirements and architecture/design are estimated and scheduled in accordance with LSST Project required budgets and schedules.

 \subsubsection{Communications} 
 A mailing list\footnote{\url{lsst-dmlt@listserv.lsstcorp.org}} exists for dmlt related messages. 
 On Mondays the DMLT hold a brief telecon(30 to 45 minutes) , this telecon serves to :
\begin{itemize}
\item Allow the Project manager and DM Scientist  to pass on important project level information and general guidance. 
\item Raise any blocking or not well understood issues across DM - this may result in calling a splinter meeting to further discuss with relevant parties.
\item Inform everyone one of any LCRs in process at LSST level which may be of interest to or  have impact on DM
\item Check on outstanding actions on DMLT members. 
\end{itemize}

Face to Face meetings of DM are held two times a year these are opportunities to:
\begin{itemize}
\item Discuss detailed planning for the next cycle
\item Discuss technical topics in a face to face environment
\item Work together on critical issues
\item Help make DM function as a team
\end{itemize}


\subsection{DM Configuration Control Board \label{sect:tct}}
The DMCCB has responsibility for issues similar to those of the LSST Configuration Control Board, but restricted to those contained within the DM subsystem. The DMCCB reviews and approves changes to all baselines in the LSST Data Management System, including proposed changes to the DM System Requirements' (DMSR), reference design, sizing model, i.e. any LDM-xxx baseline document.  The DMCCB makes sure these changes don't get into the baseline without proper change control.  Note that the DMCCB does not author the Technical Baseline and has no specific technical deliverable charter, but it does validate that the form and content of the Technical Baseline is consistent with LSST project standards such as the System Engineering Management Plan (SEMP).  Specific responsibilities for development of the Technical Baseline and evaluation of the content versus LSST and DM requirements are elsewhere in this document.
\begin{itemize}
\item Charter/purpose
	\begin{itemize}
	\item Ensure that the DM Technical Baseline (LDM-xxx) documents are baseline and once baselined only changed when necessary, according to LSST and DM configuration control processes
	\end{itemize}
\item Membership
	\begin{itemize}
	\item Chaired by the System Engineer
	\item Members include the DM System Architect, DM System Interfaces Scientist, DM SQuaRE Technical Manager and DM Project Manager
	\item For on-line virtual meetings, if a quorum is not reached within one week, the DM Project Manager will make a unilateral decision
	\end{itemize}
\item Responsibilities
	\begin{itemize}
	\item Determines when specification and deliverables are of sufficient maturity and quality to be baselined (placed under configuration controlled status) or released. The DMCCB reviews and approves proposed changes to baselined items.
	\item Reviews and approves/rejects proposed changes to baselined items
	\end{itemize}
\end{itemize}

