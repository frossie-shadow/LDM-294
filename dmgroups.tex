\section{Data Management Groups/Bodies} \label{sect:groups}

Since the DM team is distributed in terms of geography and responsibility across the LSST partner and lead institutions, mechanisms are needed to ensure that the project remains on track at all times. There are five primary coordinating bodies to ensure the management, technical, and quality integrity of the DM Subsystem.

\subsection{System Science Team \label{sect:dmsst}}

Members of the DM System Science Team (SST) work together to define, maintain, and communicate to the DM Systems Engineering team a coherent vision of the LSST DM system responsive to the overall LSST Project goals, as well as scientifically validate the as-built system (\citeds{LDM-503}, Section~9.).

\subsubsection{Organization and Goals}

The System Science Team includes:
\begin{itemize}
\item DM Subsystem Scientist (chair)
\item DM Science Validation Scientist
\item DM Institutional Science Leads
\item DM System Science Analysts
\item DM Science Pipelines Scientist
\end{itemize}

The System Science Team has been chartered to:
\begin{itemize}
\item Support the DM Subsystem Scientist (as the overall DM Product Owner) in ensuring that Data Management Subsystem's initiatives provide solutions that meet the overall LSST science goals.
\item Support the Institutional Science Leads in their roles as Product Owners for elements of the DM system their respective institutions have been tasked to deliver.
\item Support the DM Science Validation Scientist, who organizes and coordinates the science validation efforts (\citeds{LDM-503}).
\item Guide the work of System Science Analysts, who generally lead and/or execute studies needed to support SST work.
\item Provide a venue for communication with the Science Pipelines Scientist, who broadly advises on topics related to the impact of science pipelines on delivered science and vice versa (\secref{role:pipe}).
\end{itemize}

The members of the System Science Team report to the DM Subsystem Scientist and share the following responsibilities:
\begin{itemize}
\item Communicate with the science community and internal stakeholders to understand their needs, identifying the aspects to be satisfied by the DM Subsystem.
\item Liaise with the science collaborations to understand and coordinate any concurrent science investigations relevant to the DM Subsystem.
\item Develop, maintain, and articulate the vision of DM-delivered LSST data products and services that is responsive to stakeholder needs, balanced across science areas, well motivated, and scientifically and technologically current.
\item Work with the DM Project Manager and DM Technical Managers to communicate and articulate the DM System vision and requirements to the DM engineering team.
\item Identify, develop, and champion new scientific opportunities for the LSST DM System, as well as identify risks where possible.
\item Develop change proposals and/or evaluate the scientific impact of proposed changes to DM deliverables driven by schedule, budget, or other constraints.
\item Lead the Science Verification of the deliverables of the DM subsystem.
\end{itemize}

\subsubsection{Communications}

DM System Science Team communication mechanisms are described on the SST Confluence page at \url{http://ls.st/sst}.

\subsection{DM Systems Engineering Team \label{sect:sysengt}}

The Systems Engineering Team is led by the DMPM (\secref{role:dmpm}) and looks after all aspects of systems engineering.
It is comprised of not only the Systems Engineer (\secref{role:sysengineer}) but also the Software Architect (\secref{role:softarc}), Operations Architect (\secref{role:opsarc}), Pipeline Scientist (\secref{role:pipe}) and Interface Scientist (\secref{role:dmis}).

While the product owners (\secref{role:prodo}) help DM to create products which are fit for purpose, the Systems Engineering Team must ensure we do it correctly. This group concerns itself with (sub)system wide decisions on architecture and software engineering.

The specific tasks of this group include:

\begin{itemize}
\item Formalize the product list for DM\footnote{In this sense, ``products'' are the software and systems which produce data products, rather than the data products themselves. See also \ref{sect:products}.}
\item Formalize the documentation tree for DM, defining which documents need to be produced for each product
\item Agree the process for tracing the baseline requirements verification and validation status.
\item Agree the formal versions of documents and software which form the technical baseline, individual items will go through the CCB for formal approval.
\end{itemize}

Some of these tasks are will be delegated to individual group members.
These individuals also are the conduit to/from the rest of the DM team to raise ideas/issues with the engineering approach.

\subsubsection{Communications}

The Systems Engineering Team will only physically meet to discuss specific topics: there will not be a regular meeting of the group outside of the one to one meetings with the DM project manager for the individuals in the group.
Discussions will be held via email until in person talks are required.

\subsection{DM Leadership Team \label{sect:dmlt}}

The purpose of the DM Leadership Team (DMLT) is to assist the DMPM  establish the scope of work and resource allocation across DM and ensure overall project management integrity across DM.
The following mandate established the DMLT:

\begin{itemize}
\item Charter/purpose
	\begin{itemize}
	\item Maintain scope of work and keep within resource allocation across DM
	\item Ensure overall project management integrity across DM
	\item Ensure Earned Value management requirements are met
	\end{itemize}
\item Membership
	\begin{itemize}
	\item Co-chaired by the DM Project Manager (\secref{role:dmpm}) and DM Project Scientist (\secref{role:dmps})
	\item Lead Institution Technical/Control Account Managers (T/CAMs; \secref{role:tcam})
	\item Institutional Science or Engineering Leads (\secref{role:scilead})
	\item Members of the DM Systems Engineering Team (\secref{sect:sysengt})
	\end{itemize}
\item Responsibilities
	\begin{itemize}
	\item Prepares all budgets, schedules, plans
	\item Meets every week to track progress, address issues/risks, adjust work assignments and schedules, and disseminate/discuss general PM communications
	\end{itemize}
\end{itemize}

The DM Leadership Team and the DM Systems Engineering Team (\secref{sect:sysengt}) work in synchrony.
The DMLT makes sure the requirements and architecture/design are estimated and scheduled in accordance with LSST Project required budgets and schedules.

 \subsubsection{Communications}
A mailing list\footnote{\url{lsst-dmlt@listserv.lsstcorp.org}} exists for DMLT related messages.
On Mondays the DMLT hold a brief (30 to 45 minutes) telecon. This serves to:

\begin{itemize}
\item Allow the Project manager and DM Scientist  to pass on important project level information and general guidance.
\item Raise any blocking or priority issues across DM --- this may result in calling a splinter meeting to further discuss with relevant parties.
\item Inform all team members of any change requests (LCRs) in process at LSST level which may be of interest to or have an impact on DM
\item Check on outstanding actions on DMLT members
\end{itemize}

Face to Face meetings of DM are held twice a year; these are opportunities to:

\begin{itemize}
\item Discuss detailed planning for the next cycle
\item Discuss technical topics in a face to face environment
\item Work together on critical issues
\item Help make DM function as a team
\end{itemize}

\subsection{DM Change Control Board \label{sect:dmccb}}

The DMCCB has responsibility for issues similar to those of the LSST Change Control Board, but with its scope restricted to the DM Subsystem.
The DMCCB reviews and approves changes to all baselines in the Subsystem, including proposed changes to the DM System Requirements (DMSR), reference design, sizing model, i.e. any LDM-series document.
The Technical Baseline, including software/hardware and documentation, is written by DM and controlled by the DMCCB.
DMCCB validates that the form and content of the Technical Baseline is consistent with LSST project standards such as the Systems Engineering Management Plan (SEMP) \citeds{LSE-17}.

\begin{itemize}
\item Charter/purpose
	\begin{itemize}
	\item Ensure that the DM Technical Baseline (LDM-xxx) documents are baselined and subsequently changed only when necessary and according to LSST and DM configuration control processes
	\end{itemize}
\item Membership
	\begin{itemize}
	\item Chaired by the Systems Engineer
	\item Members include the DM Software Architect (\S\ref{role:softarc}), DM Operations Architect (\S\ref{role:opsarc}), DM System Interfaces Scientist (\S\ref{role:dmis}), DM SQuaRE T/CAM (\S\ref{role:tcam}), DM Release Manager (\S\ref{role:dmrm}) and DM Project Manager (\S\ref{role:dmpm})
	\item For on-line virtual meetings, if a consensus or quorum or is not reached within one week, the DM Project Manager will make a unilateral decision
	\end{itemize}
\item Responsibilities
	\begin{itemize}
	\item Determines when specification and deliverables are of sufficient maturity and quality to be baselined (placed under configuration controlled status) or released.
	\item Reviews and approves/rejects proposed changes to baselined items
	\end{itemize}
\end{itemize}

\subsection{DM Science Validation Team}
\label{sect:dmsvt}

The DM Science Validation Team guides the definition of, and receives the products of, science validation and dress rehearsal activities, following the long-term roadmap described in \citeds{LDM-503}.
Decisions on the strategic goals of these activities are made in conjunction with the DM Subsystem Scientist and Project Manager.

The DM Science Validation Team is chaired by the DM Science Validation Scientist (\secref{role:dmsvs}).
Its membership includes the DM Pipelines Scientist (\secref{role:pipe}) and the various Institutional Science/Engineering Leads (\secref{role:scilead}).
Depending on the activities currently being executed, other members of the System Science Team (\secref{sect:dmsst}), the wider DM Construction Project, and/or external experts may be temporarily added to the team.
