\section{Data Management Conceptual Architecture \label{sect:dmarc}}

The DM Subsystem Architecture is detailed in \citeds{LDM-148}.
A few of the higher level diagrams are reproduced here to orientate the reader within DM.

During Operations, components of the DM Subsystem will be installed and run in
multiple locations. These include:

\begin{itemize}
\item The Commissioning Cluster, which may be physically at NCSA in Urbana-Champaign
\item The main centre in NCSA enclave in Urbana-Champaign
\item The US Data Access Centre (DAC), also at NCSA in Urbana-Champaign
\item The Chilean DAC in the Base Facility in La Serena Chile
\item The Satellite Processing Centre at CC-IN2P3 in Lyon, France
\end{itemize}

\figref{fig:dmsdeploy} shows the various DM components which will be used in operations and the physical compute environments in which they will be deployed.
Bulk data storage and transport between components is provided by the Data Backbone. This complex piece of infrastructure is displayed in \figref{fig:databb}.

Science users will access the data products produced by LSST through the
Science Platform, as shown in \figref{fig:sciplat}.

\figref{fig:dcs} shows the common infrastructure and services layer which underlies the compute environments.
This does not list specific technologies for management/monitoring, provisioning/deployment, or workload/workflow --- these are still under development --- but consider industry-standard tools such as Nagios, Puppet/vSphere/OpenStack/Kubernetes, and Pegasus.

\begin{figure}[htbp]
\begin{center}
\includegraphics[width=0.8\textwidth]{images/DMSDeployment}
\caption{DM components as deployed during Operations. Where components are
deploed in multiple locations, the connections between them are labelled with
the relevant communication protocols. Science payloads are shown in blue.
\label{fig:dmsdeploy}}
\end{center}
\end{figure}

\begin{figure}[htbp]
\begin{center}
\includegraphics[width=0.5\textwidth]{images/SciencePlatform}
\caption{The sub-components of the Science Platform. \label{fig:sciplat}}
\end{center}
\end{figure}


\begin{figure}[htbp]
\begin{center}
\includegraphics[width=0.6\textwidth]{images/DataBackbone}
\caption{The Data Backbone links all the physical components of DM. \label{fig:databb}}
\end{center}
\end{figure}

\begin{figure}[htbp]
\begin{center}
 \includegraphics[width=0.7\textwidth]{images/DMSCommonServices}
\caption{Common infrastructure services available at each DM location. \label{fig:dcs}}
\end{center}
\end{figure}



\subsection{External Interfaces}
The DM external interfaces are controlled by the following ICDs:
\begin{itemize}
	\item[\citeds{LSE-68}] Data Acquisition Interface between Data Management and Camera
	\item[\citeds{LSE-69}] Interface between the Camera and Data Management	 
	\item[\citeds{LSE-72}] OCS Command Dictionary for Data Management
	\item[\citeds{LSE-75}] Control System Interfaces between the Telescope and Data Management
	\item[\citeds{LSE-76}] Infrastructure Interfaces between Summit Facility and Data Management
	\item[\citeds{LSE-77}] Infrastructure Interfaces between Base Facility and Data Management
	\item[\citeds{LSE-130}] List of Data Items to be Exchanged Between the Camera and Data Management
	\item[\citeds{LSE-131}] Data Management Interface Requirements to Support Education and Public Outreach 
	\item[\citeds{LSE-140}] Auxiliary Instrumentation Interface between Data Management and Telescope
\end{itemize}

\subsubsection{Auxiliary data in DM}
Certain tasks in DM rely on external catalogues and other information. Currently we believe we need:
\begin{enumerate}
		        \item Gaia catalogue (Release 2) as a photometry baseline.
		\end{enumerate}
