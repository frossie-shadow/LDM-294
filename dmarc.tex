\section{Data Management Conceptual Architecture} \label{sect:dmarc}

The DM architecture is detail in \citeds{LDM-148} - a few of the higher level diagrams are reproduced here to 
orientate the reader in DM.

In operations components of DM are installed and running in several locations. Namely:
\begin{itemize}
\item Commissioning cluster (which may be physically in NCSA)
\item The main centre, NCSA enclave in Champaign Urbana
\item The US Data Access Centre(DAC) also at NCSA in Champaign Urbana
\item The Chilean DAC in the Base Facility in La Serena Chile. 
\item The Satellite Processing Centre at IN2P3 in Lyons, France.
\end{itemize}
\figref{fig:dmsdeploy} shows these physical compute environments  and the DM components which are deployed there. 
 Some components are dotted and split across different
environments; the connections between these are labelled with the
protocols involved.  Science payloads (productions) are in shown in blue.
The Science Platform is expanded in \figref{fig:sciplat}, likewise the data backbone is 
a complex piece of infrastructure and is expanded in \figref{fig:databb}.

showing where Qserv and the Butler fit
into the system as well as metadata and provenance.




\begin{figure}[htbp]
\begin{center}
 \includegraphics[width=0.8\textwidth]{images/DMSDeployment}
\caption{This figure shows the deployment of DM components in the various DM locations \label{fig:dmsdeploy}}
\end{center}
\end{figure}

\begin{figure}[htbp]
\begin{center}
 \includegraphics[width=0.5\textwidth]{images/SciencePlatform}
\caption{This figure shows the sub components of the Science Platform. \label{fig:sciplan}}
\end{center}
\end{figure}


\begin{figure}[htbp]
\begin{center}
 \includegraphics[width=0.6\textwidth]{images/DataBackbone}
\caption{The Data backbone links all the physical components of DM. \label{fig:databb}}
\end{center}
\end{figure}

The common infrastructure services layer is underlying the compute environments. 
This does not list  specific technologies for management/monitoring,
provisioning/deployment, or workload/workflow, but you can think of them
as things like Nagios, Puppet/vSphere/OpenStack/Kubernetes, and Pegasus.

\begin{figure}[htbp]
\begin{center}
 \includegraphics[width=0.7\textwidth]{images/DMSCommonServices}
\caption{This figure shows the various Services at each DM location \label{fig:dcs}}
\end{center}
\end{figure}



\subsection{External Interfaces}
The DM external interfaces are controlled by the following ICDs:
\begin{itemize}
	\item[\citeds{LSE-68}] Data Acquisition Interface between Data Management and Camera
	\item[\citeds{LSE-69}] Interface between the Camera and Data Management	 
	\item[\citeds{LSE-72}] OCS Command Dictionary for Data Management
	\item[\citeds{LSE-75}] Control System Interfaces between the Telescope and Data Management
	\item[\citeds{LSE-76}] Infrastructure Interfaces between Summit Facility and Data Management
	\item[\citeds{LSE-77}] Infrastructure Interfaces between Base Facility and Data Management
	\item[\citeds{LSE-130}] List of Data Items to be Exchanged Between the Camera and Data Management
	\item[\citeds{LSE-131}] Data Management Interface Requirements to Support Education and Public Outreach 
	\item[\citeds{LSE-140}] Auxiliary Instrumentation Interface between Data Management and Telescope
\end{itemize}

\subsubsection{Auxiliary data in DM}
Certain tasks in DM rely on external catalogues and other information. Currently we believe we need:
\begin{enumerate}
		        \item Gaia catalogue (Release 2) as a photometry baseline.
		\end{enumerate}
