\newpage
\section{Project Controls}\label{sect:dmpc}

DM follows the LSST project controls system, as described in \citeds{LPM-98}.
Specific DM processes for project planning are elaborated in \citeds{LDM-472}.

The LSST Project Controller is responsible for the PMCS and, in particular, for ensuring that DM properly complies with our earned value management requirements.
He is the first point of contact for all questions about the PMCS system.

\subsection{Schedule  \label{sect:schedule} }
The entire LSST project schedule is held in Primavera. Tied to major project milestones we have  a series of DM tests which need to be performed to show readiness for he different project phases.
This is depicted in \figref{fig:schedule}.

\begin{figure}[htbp]
	\begin{center}
		 \includegraphics[width=\textwidth]{images/DMMasterSchedule}
		 \caption{DM major milestones(LDM-503-x) in the LSST schedule. \label{fig:schedule}}
	 \end{center}
 \end{figure}



\subsection{Work Breakdown Structure} \label{sect:WBS}

The DM WBS is laid out in \citeds{LPM-43} with definitions provided in \citeds{LPM-44},
the new WBS is currently described in \appref{sec:wbslist} making LPM-43 out of date.

The WBS provides a hierarchical index of all hardware, software, services, and other deliverables which are required to complete the LSST Project.
It consists of alphanumeric strings separated by periods.
The first component is always “1”, referring the LSST Construction Project.
``02C'' in the second component corresponds to Data Management Construction.
Subdivisions thereof are indicated by further digits.
These subdivisions correspond to teams within the DM project.
The top level WBS elements are mapped to the lead institutes in \tabref{tab:wbs}, the lead institutions roles are outlined in \secref{sect:leadtutes}.
The various groups involved in the WBS are briefly described in \secref{sect:groups}.

\begin{table}
\caption{DM top level Work Breakdown Structure \label{tab:wbs}}
\begin{tabular}[htb]{|l|l|l|} \hline
\textbf{WBS}  &  \textbf{Description}   &  \textbf{Lead Institution}\\ \hline
1.02C.01& System Management                       &  LSST Tucson \\ \hline
1.02C.02& Systems Engineering                     &  LSST Tucson \\ \hline
1.02C.03& Alert Production                        &  University of Washington\\ \hline
1.02C.04& Data Release Production                 &  Princeton University\\ \hline
1.02C.05& Science User Interface                  &  Caltech IPAC\\ \hline
1.02C.06& Science Data Archive                    &  SLAC\\ \hline
1.02C.07& Processing Control \& Site Infrastructure & NCSA\\ \hline
1.02C.08& International Communications. \& Base Site&  LSST Tucson \\ \hline
1.02C.09& Systems Integration \& Test               & NCSA \& LSST Tucson \\ \hline
1.02C.10& Science Quality \& Reliability Engineering& LSST Tucson \\ \hline
\end{tabular}
\end{table}
